Impractically surreal: an exploration of the surreal numbers using an implementation in Julia


Note that shorthand makes surreals look simpler than they are ...


Let's consider a suposedly simple case $2 \times 2$, starting with a
simple, canonical form for 2. That is compute $x \times y$ where
\begin{eqnarray*}
  x & = & \{ 1 \mid \phi \} \\
  y & = & \{ 1 \mid \phi \} \\
\end{eqnarray*}

The definition of mulitplication is given above, but note that $X_R =
Y_R = \phi$ and so 3 of the four terms in the multiplication are also
$\phi$. Taking the only non-empty term we get
\begin{eqnarray*}
  x y & = & \{ X_L y + x Y_L - X_L Y_L \mid \phi \} \\
      & = & \{ 1.2 + 2.1 - 1.1 \mid \phi \} \\
      & = & \{ 2 + 2 - 1 \mid \phi \}.
\end{eqnarray*}
the last simplification arising because 1 is the multiplicative identity. 

Superficially, we can complete the arithmetic above to find that $x y
= \{ 3 | \phi \}$, which would be the canonical form of 4, the
expected solution. However, we should understand that the additions
and subtraction are surreal operators, and should be evaluated as
such. Each of the integers in the computation is in canonical form (by
construction), and so $2 + 2 = 4 = \{ 3 \mid \phi \}$, which is an
easy start. However, we write the subtraction as
\begin{eqnarray*}
  \{ 3 \mid \phi \} - \{ 1 \mid \phi \}
  & = &   \{ 3 \mid \phi \} +  -\{ 1 \mid \phi \} \\
  & = &   \{ 3 \mid \phi \} +  \{ \phi \mid -1 \} \\
  & = &   \{ 2 \mid 4 \}.
\end{eqnarray*}
Note that this is equal to 3 as required, but that it is not
canonical, and in fact doesn't even seem to be simplifying (it is, by
definition, but this is part of issue here).

Note that once again, in writing $\{ 2 \mid 4\}$ we have simplified
the arithmetic down to it's shorthand. The '4' here is canonical, but
the 2 is not; it is really $3 - 1$, which simillary to $4-1$ above
breaks down into two components again. And so on.

In the end, the true representation of 2.2, which we can see through
looking at it as a tree, has 21 nodes (the original surreal, and 20
component surreals), reaching down branches of length 6 to the leaves
of the tree which are the zeros,

PICTURE

A tree with 21 nodes doesn't seem so bad, perhaps. But it quickly gets
worse. The table below shows the size (in terms of numbers of nodes)
of the tree-representation of several products.

product &  size & tree depth  & number of leaves \\
2.0 &           1 &  0 & 0
2.1 &           3 &  2 & 1
2.2 &          21 &  6 & 5
2.3 &       2,574 & 12 & 652 
2.4 &   3,042,201 & 20 & 748,077
2.5 &         ?   &  ? &   ?

To illustrate the point 2.3 is shown below:

PICTURE

So far, I haven't succeeded in calculting 2.5 let alone 3.3. My code
will have to be a good deal more efficient to calculate the next
step. Guessing, I'd say that the growth from 2.1 to 2.2 was about a
factor of 10, then from 2.2 to 2.3 was about a factor of 100, and from
2.3 to 2.4 about 1000, so we might expect the next step to be about
10^4 x 10^6, so of the order of 10^{10} nodes. But computation time is
non-linear in the numb er of nodes as many of the operations that we
take for granted as constant time are (in this case) implemented
recuseively, and hence linear in the size of the trees. That is, even
constructing a surreal (as part of the tree) will require a check that
$X_L < X_R$ and hence each piece we try to construct will take time
linear in its subtree, and hence the overall compute time will be
non-linear in the size of the tree.



Tree depth seems to grow by the next even integer, i.e., 2, 4, 6, 8,
so I hypothesize that 2.5 will have tree depth 30.



We can always cheat, and recanonicalise at each step. But that is no
more than just doing traditional arithemetic.


Does it matter -- I suspect not. The motivations for surreal numbers
are not related to replacing traditional computation. ...






--- canonical form means all components are also in canonical form ...





